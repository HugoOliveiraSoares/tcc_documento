\section{Objetivos}
\subsection{Objetivo geral}

Este texto já se encontra no padrão de espaçamento correto. Ao elaborar o seu conteúdo, verificar se o espaçamento entre linhas é de 1,5 e se há uma linha livre entre parágrafos com espaçamento 0 pt antes e depois. 

Indique de forma genérica qual objetivo deve ser alcançado. Está ligado a uma visão global e abrangente do tema. Relaciona-se com o conteúdo geral a ser apresentado como resultado, do problema de pesquisa.   

\subsection{Objetivos específicos}

Este texto já se encontra no padrão de espaçamento correto. Ao elaborar o seu conteúdo, verificar se o espaçamento entre linhas é de 1,5 e se há uma linha livre entre parágrafos com espaçamento 0 pt antes e depois. 

Têm função intermediária e instrumental, aplicado a situações particulares. São desdobramentos do objetivo geral, como as etapas a serem cumpridas para atingir o mesmo. Apresentam caráter mais concreto. Para construir este tópico, responda: “para quem fazer”? 

Não é uma regra, mas em geral são os capítulos e seus tópicos na composição do trabalho. 

Lembre-se:  
\begin{itemize}

 \item Usar verbos no infinitivo, tais como: verificar, avaliar, identificar, explicar, etc. 

 \item Servem para responder à pergunta: “O QUÊ?” 

 \item Objetivo é sinônimo de meta, fim. Indica o que o pesquisador quer atingir com o trabalho de pesquisa (o que você quer fazer, que metas você quer alcançar).  

 \item Devem ser apresentados em tópicos 
\end{itemize}

\cite{OpenVLC}
