\documentclass[12pt,article,a4paper,brazil]{abntex2}

\usepackage[brazil]{babel}
\usepackage[left=3.0cm,top=3.0cm,right=2.0cm,bottom=2.0cm]{geometry}
\usepackage{indentfirst}
\usepackage{graphicx}
\usepackage{xcolor}
\usepackage[alf]{abntex2cite}	% Citações padrão ABNT
\usepackage{fontspec}
\setmainfont{Arial}

\setlength{\parindent}{1.25cm}
\linespread{1.5}

\hypersetup{
  		colorlinks=true,
    	citecolor=black,
      linkcolor=black,
      urlcolor=blue 
}

\addto\captionsbrazil{
%% ajusta nomes padroes do babel
\renewcommand{\bibname}{Referências}
\renewcommand{\indexname}{Índice}
\renewcommand{\listfigurename}{Lista de ilustrações}
\renewcommand{\listtablename}{Lista de tabelas}
%% ajusta nomes usados com a macro \autoref
\renewcommand{\pageautorefname}{página}
\renewcommand{\sectionautorefname}{seção}
\renewcommand{\subsectionautorefname}{subseção}
\renewcommand{\paragraphautorefname}{parágrafo}
\renewcommand{\subsubsectionautorefname}{subseção}
}


\data{2023}

\begin{document}

  
  \titulo{VLC(Visible Light Communication)}
  \autor{Hugo Oliveira Soares}
  \local{Belo Horizonte}
  \instituicao{Dom Helder Escola Superior}
  \orientador{Prof. XXX}
  \coorientador{Prof. Ricardo Luiz de Freitas}
  \preambulo{
    Projeto de Pesquisa apresentado à Escola de Engenharia de Minas Gerais – EMGE como requisito parcial para obtenção do título de Cientista da Computação.
    \par
    Orientador(a) de conteúdo: \imprimirorientador
    \par
    Orientador de metodologia: \imprimircoorientador
  }

  % Parte pré-textual
  \renewcommand{\imprimircapa}{
  \begin{capa}
    \center
    \begin{figure}[ht]
        \begin{flushright}
          \includegraphics[scale=0.8]{images/logoDom.png}
          \label{Logo Dom Helder Escola Superior}
        \end{flushright}
    \end{figure}

    {\ABNTEXchapterfont\Large\imprimirinstituicao}

    \vspace*{2cm}

    {\ABNTEXchapterfont\large\imprimirautor}

    \vfill
    \begin{center}
      {\ABNTEXchapterfont\bfseries\LARGE\imprimirtitulo}
    \end{center}
    \vfill

   
    \imprimirlocal
    \par
    \imprimirdata

    % \vspace*{1cm}
  \end{capa}
}


 
  \makeatletter
\renewcommand{\folhaderostocontent}{
  \begin{center}

    \vspace*{\fill}
    \vspace*{\fill}
    \begin{center}
      \ABNTEXchapterfont\bfseries\Large\imprimirtitulo
    \end{center}

    \vspace*{\fill}

    \abntex@ifnotempty{\imprimirpreambulo}{
      \hspace{.45\textwidth}
      \begin{minipage}{.5\textwidth}
        \SingleSpacing
          \ABNTEXfontereduzida
          \imprimirpreambulo
      \end{minipage}
      \vspace*{\fill}
    }


    \vspace*{\fill}

    \imprimirlocal
    \par
    \imprimirdata
    \vspace*{1cm}

    \end{center}
}
\makeatother

  \imprimircapa
  \imprimirfolhaderosto

  % \listoffigures
  % \listoftables
  \tableofcontents %SUMARIO

  %% Capitulos
  \section{Introdução}

Este texto já se encontra no padrão de espaçamento correto. Ao elaborar o seu conteúdo, verificar se o espaçamento entre linhas é de 1,5 e se há uma linha livre entre parágrafos com espaçamento 0 pt antes e depois.

Pode conter o tema, a justificativa, as questões ou hipóteses – formulação, delimitação, problema específico a resolver quanto ao tema no decorrer da pesquisa (o que será pesquisado; a “pergunta”); justificativa, segundo os critérios habituais: relevância; originalidade (a necessidade de incluir este item na pesquisa normalmente exigirá uma exposição do que já foi feito acerca do tema, ou de temas próximos, no contexto da disciplina em que a pesquisa se desenvolve, ou de disciplinas diferentes, mas pertinentes ao tema em questão); viabilidade; interesse pessoal (por que escolheu esse problema); e objetivos (estes podem ser apresentados em item separado, após a introdução ou no texto da introdução). A redação dos objetivos deve ser extremamente breve quanto ao que se pretende obter nos diversos níveis que forem pertinentes para a realização da pesquisa em questão. Tal exposição deve ser inteligível mesmo para pessoas não especializadas na disciplina em cujo contexto se formula e se realiza a pesquisa. 

  \section{PRESSUPOSTOS/ HIPÓTESES/PROBLEMÁTICA (OPCIONAL)} 
Este texto já se encontra no padrão de espaçamento correto. Ao elaborar o seu conteúdo, verificar se o espaçamento entre linhas é de 1,5 e se há uma linha livre entre parágrafos com espaçamento 0 pt antes e depois. 

Hipótese é sinônimo de suposição e premissas. É a tese a ser defendida no trabalho e, por ter tal característica de “possibilidade” de resposta, no final da pesquisa ela poder ser confirmada ou refutada.  

É destinada a explicar provisoriamente um problema até que os fatos venham a contradizê-la ou confirmá-la. É uma proposição testável que pode vir a ser a solução do problema.\newline
Portanto, de forma muito simples, é a “resposta provisória” para o problema de pesquisa formulado em relação ao tema (apresentado no tópico anterior).   

  \section{Objetivos}
\subsection{Objetivo geral}

Este texto já se encontra no padrão de espaçamento correto. Ao elaborar o seu conteúdo, verificar se o espaçamento entre linhas é de 1,5 e se há uma linha livre entre parágrafos com espaçamento 0 pt antes e depois. 

Indique de forma genérica qual objetivo deve ser alcançado. Está ligado a uma visão global e abrangente do tema. Relaciona-se com o conteúdo geral a ser apresentado como resultado, do problema de pesquisa.   

\subsection{Objetivos específicos}

Este texto já se encontra no padrão de espaçamento correto. Ao elaborar o seu conteúdo, verificar se o espaçamento entre linhas é de 1,5 e se há uma linha livre entre parágrafos com espaçamento 0 pt antes e depois. 

Têm função intermediária e instrumental, aplicado a situações particulares. São desdobramentos do objetivo geral, como as etapas a serem cumpridas para atingir o mesmo. Apresentam caráter mais concreto. Para construir este tópico, responda: “para quem fazer”? 

Não é uma regra, mas em geral são os capítulos e seus tópicos na composição do trabalho. 

Lembre-se:  
\begin{itemize}

 \item Usar verbos no infinitivo, tais como: verificar, avaliar, identificar, explicar, etc. 

 \item Servem para responder à pergunta: “O QUÊ?” 

 \item Objetivo é sinônimo de meta, fim. Indica o que o pesquisador quer atingir com o trabalho de pesquisa (o que você quer fazer, que metas você quer alcançar).  

 \item Devem ser apresentados em tópicos 
\end{itemize}

\cite{OpenVLC}

  \section{Justificativa}

Este texto já se encontra no padrão de espaçamento correto. Ao elaborar o seu conteúdo, verificar se o espaçamento entre linhas é de 1,5 e se há uma linha livre entre parágrafos com espaçamento 0 pt antes e depois. 
\cite{conceiccao2015comunicaccao}

\textit
A justificativa responde à pergunta “POR QUÊ?” 

Como o próprio nome indica, é o convencimento de que o trabalho de pesquisa deve ser efetivado.  

Apresente a relevância técnica, científica e socialmente sua proposta. Explicite argumentos que indiquem que sua pesquisa é significativa, importante ou relevante.  

Para ajudar, tente pensar nos três itens que não podem deixar de ser observados na justificativa. 

a) IMPORTÂNCIA: Que revela o porquê de se estudar tal tema. Por que o estudo desse tema é importante para a área em questão (Inteligência Artificial, por exemplo) e importante para você (pesquisador)? Aqui se concentra a chamada justificativa científica.  

b) VIABILIDADE: Quais são as possibilidades de se realizar esta pesquisa? Este aspecto está relacionado às possibilidades materiais da pesquisa: fontes de consulta disponíveis, etc. 

c) OPORTUNIDADE: Por que esta pesquisa é oportuna neste momento? Ela está de acordo com os interesses da atualidade? Aqui se concentra a chamada justificativa social-científica, que demonstra contribuição de seu conhecimento para a sociedade. 

Portanto, podem ser justificativas de ORDEM PESSOAL (relacionadas aos interesses dos pesquisadores, experiência ou possibilidade de atuação na área selecionada), de ORDEM TÉCNICA (acesso ao material e fontes de pesquisa, como livros, estatísticas, informações sobre a empresa) e ORDEM CIENTÍFICA (com a contribuição para a área do conhecimento, por ser um tema novo ou já existente e não satisfatoriamente respondido na área acadêmica, com espaço para novos debates). 

  \section{REFERENCIAL TEÓRICO}  

\textcolor{red}{(apresentar 1 linha livre)}

Este texto já se encontra no padrão de espaçamento correto. Ao elaborar o seu conteúdo, verificar se o espaçamento entre linhas é de 1,5 e se há uma linha livre entre parágrafos com espaçamento 0 pt antes e depois. 

Expor resumidamente as principais ideias já discutidas por outros autores que trataram do problema, levantando críticas e dúvidas, quando for o caso. Explicar no que seu trabalho vai se diferenciar dos trabalhos já produzidos sobre o problema a ser trabalhado e/ou no que vai contribuir para seu conhecimento. Quanto ao quadro teórico, o erro mais frequente é formulá-lo de forma genérica ou abstrata demais, quando o que interessa é que ele seja adequado ao recorte temático a ser investigado; quanto à formulação das hipóteses ou das questões, não basta enunciá-las no projeto, é preciso também justificá-las uma a uma em texto argumentativo. 

\noindent
\textcolor{red}{FIGURAS E FOTOS - Padrão: Júnia Lessa França (2019) 10ª ed. – pág 106 
\\
GRÁFICOS - Padrão: Júnia Lessa França (2019) 10ª ed. – pág 111 
\\
TABELAS E QUADROS - Padrão: Júnia Lessa França (2019) 10ª ed. – pág 111 }

  \section{METODOLOGIA}
\textcolor{red}{(apresentar 1 linha livre)}

Este texto já se encontra no padrão de espaçamento correto. Ao elaborar o seu conteúdo, verificar se o espaçamento entre linhas é de 1,5 e se há uma linha livre entre parágrafos com espaçamento 0 pt antes e depois. 
\cite{reforma}

Esclarecer se a pesquisa é de natureza básica ou aplicada e, quanto aos objetivos, se é exploratória, descritiva ou explicativa. Indicar também o procedimento a ser adotado: pesquisa experimental, levantamento, estudo de caso, pesquisa bibliográfica, ou outro. 

Definir o universo de estudo e os critérios de inclusão e exclusão do processo de amostragem. 

Descrever as técnicas utilizadas para a coleta de dados e os instrumentos utilizados (de acordo com o tipo de técnica escolhida), a serem apresentados em anexo, se necessário. 

A coleta de dados é a busca por informações para a elucidação do fenômeno ou fato que o pesquisador quer desvendar. O instrumental técnico elaborado pelo pesquisador para o registro e a medição dos dados deverá preencher os seguintes requisitos: validez, confiabilidade e precisão.  

Descrever a metodologia de análise de dados e as ferramentas utilizadas para tal fim. 


  \bibliography{referencias}

\end{document}
