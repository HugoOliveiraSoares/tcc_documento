\section{Justificativa}

Este texto já se encontra no padrão de espaçamento correto. Ao elaborar o seu conteúdo, verificar se o espaçamento entre linhas é de 1,5 e se há uma linha livre entre parágrafos com espaçamento 0 pt antes e depois. 
\cite{conceiccao2015comunicaccao}

\textit
A justificativa responde à pergunta “POR QUÊ?” 

Como o próprio nome indica, é o convencimento de que o trabalho de pesquisa deve ser efetivado.  

Apresente a relevância técnica, científica e socialmente sua proposta. Explicite argumentos que indiquem que sua pesquisa é significativa, importante ou relevante.  

Para ajudar, tente pensar nos três itens que não podem deixar de ser observados na justificativa. 

a) IMPORTÂNCIA: Que revela o porquê de se estudar tal tema. Por que o estudo desse tema é importante para a área em questão (Inteligência Artificial, por exemplo) e importante para você (pesquisador)? Aqui se concentra a chamada justificativa científica.  

b) VIABILIDADE: Quais são as possibilidades de se realizar esta pesquisa? Este aspecto está relacionado às possibilidades materiais da pesquisa: fontes de consulta disponíveis, etc. 

c) OPORTUNIDADE: Por que esta pesquisa é oportuna neste momento? Ela está de acordo com os interesses da atualidade? Aqui se concentra a chamada justificativa social-científica, que demonstra contribuição de seu conhecimento para a sociedade. 

Portanto, podem ser justificativas de ORDEM PESSOAL (relacionadas aos interesses dos pesquisadores, experiência ou possibilidade de atuação na área selecionada), de ORDEM TÉCNICA (acesso ao material e fontes de pesquisa, como livros, estatísticas, informações sobre a empresa) e ORDEM CIENTÍFICA (com a contribuição para a área do conhecimento, por ser um tema novo ou já existente e não satisfatoriamente respondido na área acadêmica, com espaço para novos debates). 
