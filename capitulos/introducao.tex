\section{Introdução}

Este texto já se encontra no padrão de espaçamento correto. Ao elaborar o seu conteúdo, verificar se o espaçamento entre linhas é de 1,5 e se há uma linha livre entre parágrafos com espaçamento 0 pt antes e depois.

Pode conter o tema, a justificativa, as questões ou hipóteses – formulação, delimitação, problema específico a resolver quanto ao tema no decorrer da pesquisa (o que será pesquisado; a “pergunta”); justificativa, segundo os critérios habituais: relevância; originalidade (a necessidade de incluir este item na pesquisa normalmente exigirá uma exposição do que já foi feito acerca do tema, ou de temas próximos, no contexto da disciplina em que a pesquisa se desenvolve, ou de disciplinas diferentes, mas pertinentes ao tema em questão); viabilidade; interesse pessoal (por que escolheu esse problema); e objetivos (estes podem ser apresentados em item separado, após a introdução ou no texto da introdução). A redação dos objetivos deve ser extremamente breve quanto ao que se pretende obter nos diversos níveis que forem pertinentes para a realização da pesquisa em questão. Tal exposição deve ser inteligível mesmo para pessoas não especializadas na disciplina em cujo contexto se formula e se realiza a pesquisa. 
